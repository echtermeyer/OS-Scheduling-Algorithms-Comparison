% !TEX root =  master.tex

\section{Innerhalb von Betriebssystemen}
Ein häufiges Anwendungsgebiet von \ac{FCFS} ist für Aufgaben der Batch-Verarbeitung, da hierbei der Convoy-Effekt nicht entscheidend ist. Ein klassisches Beispiel ist das Druckmanagement in frühen Betriebssystemen, wo Druckaufträge in der Reihenfolge ihres Eintreffens abgearbeitet werden.  %\textit{Referenz:} Silberschatz, Galvin, and Gagne. \textit{Operating System Concepts}. 9th ed., John Wiley \& Sons, 2012.

Round Robin hingegen ist ein weit verbreiteter Scheduling Algorithmus in zeitkritischen Betriebssystemen. Ein Beispiel hierfür ist das Thread-Scheduling in Unix-basierten Systemen, bei dem jedem Thread eine feste Zeitscheibe zugeteilt wird. %\textit{Referenz:} Tanenbaum, Andrew S., and Bos, Herbert. \textit{Modern Operating Systems}. 4th ed., Pearson, 2014.

\ac{MLQ} wird in modernen Betriebssystemen wie Linux angewendet, um Prozesse basierend auf ihrer Priorität zu ordnen, wobei systemkritische Prozesse höher priorisiert werden als Benutzerprozesse. % \textit{Referenz:} Love, Robert. \textit{Linux Kernel Development}. 3rd ed., Addison-Wesley Professional, 2010.
