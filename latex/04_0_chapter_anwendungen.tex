% !TEX root =  master.tex
\chapter{Anwendungsgebiete}
In diesem Kapitel wird auf konkrete Anwendungsgebiete der oben genannten OS-Scheduling-Algorithmen eingegangen. Da diese Algorithmen auch außerhalb von Betriebssystemen eine zentrale Rolle spielen, werden auch Beispiele für solche Anwendungsgebiete aufgenommen.

In modernen Betriebssystemen ist der Aufbau der Prozessverwaltung und die Anwendung von Scheduling-Algorithmen ein komplexes Thema, das in der Praxis eine Vielzahl von Anforderungen erfüllen muss. Die hier vorgestellten Algorithmen sind nur ein kleiner Ausschnitt aus der Vielzahl von Scheduling-Algorithmen, die in modernen Betriebssystemen zum Einsatz kommen.

\subsubsection{Windows}
Windows verwendet beispielsweise 7 Prioritätsstufen.
Prozesse können sich selbst die folgenden Stufen zuweisen:
\begin{multicols}{3}
    \begin{itemize}[noitemsep]
        \item IDLE
        \item BELOW NORMAL
        \item NORMAL
        \item ABOVE NORMAL
        \item HIGH
        \item REALTIME
    \end{itemize}
\end{multicols}

Innerhalb eines Prozesses können die einzelnen Threads dann jeweils 7 Prioritätsebenen haben, die die Threads untereinander sortieren:
\begin{multicols}{3}
    \begin{itemize}[noitemsep]
        \item IDLE
        \item LOWEST
        \item BELOW NORMAL
        \item NORMAL
        \item ABOVE NORMAL
        \item HIGHEST
        \item CRITICAL
    \end{itemize}
\end{multicols}

Windows verwendet beide Prioritätszuweisungen, um dem Thread eine Basispriorität zwischen 0 und 31 zuzuweisen \autocite{KarlBridgeMicrosoft.2023}.

\subsubsection{MacOS}
Mit der Einführung der neuen M-Series Prozessoren unterteilt Apple die Prozessorkerne in Leistungs- und Effizienzkerne. Die Leistungskerne sind für rechenintensive Aufgaben zuständig, die Effizienzkerne für weniger rechenintensive Aufgaben. \autocite{hoakley.2022}

Im Allgemeinen verwendet MacOS, ähnlich wie Windows, verschiedene Prioritätsstufen oder Quality of Services, die in 4 Kategorien unterteilt werden können:
\begin{multicols}{2}
    \begin{itemize}[noitemsep]
        \item Background
        \item Utility
        \item User Initiated
        \item User Interactive
    \end{itemize}
\end{multicols}
Prozesse der Priorität Background werden nur auf den Effizienzkernen ausgeführt, während Prozesse von höherer Priorität auf beiden Prozessorkernarten ausgeführt werden können.
Entwickler können das Verhalten eines Prozesses steuern und beispielsweise festlegen, dass ein User Initiated Process trotz des hohen Quality of Service nur den Effizienzkern verwenden soll.
Laut \Citeauthor{hoakley.2022b} wird innerhalb der einzelnen Prioritätsstufen eine Art \ac{FCFS} verwendet, wobei die Prozesse in der Reihenfolge ihres Eintreffens abgearbeitet werden \autocite{hoakley.2022b}.

\subsubsection{Linux}
Da Linux ein Open-Source-Betriebssystem ist, kann hier genau angegeben werden, welches Scheduling-Verfahren verwendet wird. Im Jahr 2007 wurde das sogenannte \ac{cfs} eingeführt, welches alle Prozesse möglichst fair behandeln soll.
Linux verwendet hier verschiedene Konzepte, um Fairness zu gewährleisten.
Eine Besonderheit, die hier jedoch hervorsticht, ist die Verwendung eines Rot-Schwarz-Baumes, der die Prozesse in ihrer Ausführungsreihenfolge sortiert.
Soll ein neuer Prozess hinzugefügt werden, kann dieser an der entsprechenden Stelle im Baum eingefügt werden.
Der Scheduler wählt dann den Prozess aus, der sich auf dem äußersten linken Blatt des Baumes befindet und führt diesen aus.
Der Vorteil dieses Baumes ist die Laufzeit beim Einfügen oder Entfernen eines Prozesses von $O(\log n)$, wobei $n$ die Anzahl der Prozesse ist \autocite{Jones.2009}.

Im Jahr 2023 wurde der \ac{cfs} durch den \ac{eevdf} ersetzt. Dieser ergänzt den \ac{cfs} um einige Punkte.
So wird die Zeit, die ein Prozess auf der \ac{cpu} verbringt; mit der Zeit, die er auf der \ac{cpu} verbringen sollte; verglichen. Dieser Wert und eine zusätzlich vom Prozess angegebenen Dringlichkeit werden dann bei der erneuten Einsortierung des Prozesses in die \enquote{Warteschlange} berücksichtigt.
Dadurch können dringliche Prozesse sowie Prozesse die weniger Zeit in der \ac{cpu} verbracht haben bevorzugt werden \autocite{Muller.01112023}.


\subsubsection{Außerhalb von Betriebssystemen}
\ac{FCFS} wird in zahlreichen Gebieten verwendet. \Citeauthor{Groover.2016} erklärt beispielsweise den Einsatz beim Planen von Produktionsabfolgen. Hier wird unter anderem \ac{FCFS} verwendet um von dessen Fairness zu profitieren \Autocite[S.735]{Groover.2016}.

Ein bekanntes Anwendungsgebiet von Round Robin ist die Telekommunikation. Hierbei wird Round Robin für die Paketvermittlung und Lastverteilung in Netzwerkroutern eingesetzt \Autocite[Kapitel 4.2.4]{Kurose.2010}.
