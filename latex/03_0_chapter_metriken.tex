% !TEX root =  master.tex
\chapter{Performance Analyse}
Die richtige Auswahl dieser Algorithmen ist ein kritischer Bestandteil, um die Gesamtleistung und Reaktionsfähigkeit des Computersystems zu optimieren. Die zentrale Herausforderung hierbei ist die begrenzte Leistungsfähigkeit der Hardware, insbesondere der \ac{CPU}, da diese zu einem Zeitpunkt stets nur einen Prozess ausführen kann. Dieser Engpass wird durch die OS Scheduling Algorithmen gemindert, da diese entscheiden welcher Prozess als nächstes Zugang zur \ac{CPU} erhalten soll. Um den am besten passenden OS Scheduling Algorithmus für das spezifische Anwendungsgebiet auswählen zu können, ist es hilfreich unterschiedliche Metriken zu betrachten, um die richtige Balance zu finden. Im folgenden werden die wichtigsten Metriken kurz vorgestellt und anschließend auf die bereits zuvor beschriebenen OS Scheduling Algorithmen \ac{FCFS}, Round Robin und \ac{MLQ} angewendet. 
