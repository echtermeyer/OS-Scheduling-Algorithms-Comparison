% !TEX root =  master.tex
\chapter{Performance Analyse}
Die richtige Auswahl der zuvor vorgestellten Scheduling Algorithmen ist ein essenzieller Bestandteil, um die Gesamtleistung und Reaktionsfähigkeit eines Systems zu optimieren. Die zentrale Herausforderung hierbei ist die begrenzte Verfügbarkeit der \ac{cpu}, da diese zu einem Zeitpunkt stets nur einen Prozess ausführen kann. \ac{OS} Scheduling Algorithmen versuchen den Einsatz der \ac{cpu} zu optimieren, indem diese entscheiden welcher Prozess als nächstes bearbeitet werden soll. Die Auswahl des am besten geeigneten Algorithmus muss daher gewährleistet werden \autocite{goel_comparative_2013}. Dieses Kapitel beschäftigt sich daher mit den wichtigsten Metriken zur Auswahl der Algorithmen, um im anschließenden Abschnitt \ac{FCFS}, Round Robin und \ac{MLQ} basierend auf Simulationsergebnissen miteinander zu vergleichen.