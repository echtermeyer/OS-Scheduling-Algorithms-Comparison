% !TEX root =  master.tex

\section{Außerhalb von Betriebssystemen}
\ac{FCFS} wird in zahlreichen Gebieten verwendet, insbesondere bei Prozessen die eine schwierige Digitalisierbarkeit aufweisen. Beispielsweise wird \ac{FCFS} in der Industrieautomatisierung, insbesondere in der Steuerung von Fertigungsstraßen, verwendet, bei welcher Aufträge in der Reihenfolge ihres Eingangs bearbeitet werden. % \textit{Referenz:} Groover, Mikell P. \textit{Automation, Production Systems, and Computer-Integrated Manufacturing}. 4th ed., Prentice Hall, 2016.
Auch in Bereichen fernab von Computern und Industrie wird \ac{FCFS} in alltäglichen Situationen eingesetzt. Ob die Warteschlange an der Kasse im Supermarkt, die Patientenabfertigung im Krankenhaus in der Notaufnahme oder auch der Essensausgabe in der Kantine. 

Ein bekanntes Anwendungsgebiet von Round Robin ist die Telekommunikation. Hierbei wird Round Robin für die Paketvermittlung und Lastverteilung in Netzwerkroutern eingesetzt. Ein Beispiel ist die Verteilung von Netzwerkbandbreite in Cisco-Routern. %\textit{Referenz:} Kurose, James F., and Ross, Keith W. \textit{Computer Networking: A Top-Down Approach}. 7th ed., Pearson, 2016.
Darüber hinaus wird Round Robin in der Luftfahrtindustrie für die Flugzeugbodenabfertigung eingesetzt, um eine faire und effiziente Zuteilung von Abfertigungsdiensten zu gewährleisten. %\textit{Referenz:} Wells, Alexander T., and Young, Seth B. \textit{Airport Planning \& Management}. 6th ed., McGraw-Hill Education, 2011.

\ac{MLQ} findet beispielsweise Anwendung in Cloud-Computing-Umgebungen wie AWS oder Google Cloud, wo verschiedene Instanzen oder Services in getrennten Warteschlangen basierend auf SLAs verwaltet werden. %\textit{Referenz:} Rhoton, John. \textit{Cloud Computing Explained: Implementation Handbook for Enterprises}. 2nd ed., Recursive Press, 2013.



