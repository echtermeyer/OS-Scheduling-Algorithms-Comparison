%%%%%%%%%%%%%%%%%%%%%%%%%%%%%%%%%%%%%%%%%%%%%%%%%%%%%%%%%%
%   Autoren:
%   Prof. Dr. Bernhard Drabant
%   Prof. Dr. Dennis Pfisterer
%   Prof. Dr. Julian Reichwald
%%%%%%%%%%%%%%%%%%%%%%%%%%%%%%%%%%%%%%%%%%%%%%%%%%%%%%%%%%

%%%%%%%%%%%%%%%%%%%%%%%%%%%%%%%%%%%%%%%%%%%%%%%%%%%%%%%%%%
%	ANLEITUNG: 
%   1. Ersetzen Sie firmenlogo.jpg im Verzeichnis img
%   2. Passen Sie alle Stellen im Dokument an, die mit 
%      @stud 
%      markiert sind
%%%%%%%%%%%%%%%%%%%%%%%%%%%%%%%%%%%%%%%%%%%%%%%%%%%%%%%%%%

%%%%%%%%%%%%%%%%%%%%%%%%%%%%%%%%%%%%%%%%%%%%%%%%%%%%%%%%%%
%	ACHTUNG: 
%   Für das Erstellen des Literaturverzeichnisses wird das 
%   modernere Paket biblatex in Kombination mit biber 
%   verwendet - nicht mehr das ältere Paket BibTex!
%
%   Bitte stellen Sie Ihre TeX-Umgebung entsprechend ein (z.B. TeXStudio): 
%   Einstellungen --> Erzeugen --> Standard Bibliographieprogramm: biber
%%%%%%%%%%%%%%%%%%%%%%%%%%%%%%%%%%%%%%%%%%%%%%%%%%%%%%%%%%

\documentclass[fontsize=12pt,BCOR=5mm,DIV=12,parskip=half,listof=totoc,
               paper=a4,toc=bibliography,pointlessnumbers]{scrreprt}
               
               %toc=listof,listof=entryprefix,
               
\makeindex

%% Elementare Pakete, Konfigurationen und Definitionen werden geladen (gegebenenfalls anpassen)
\input{config}

%%
%% @stud
%%
%% PERSÖNLICHE ANGABEN (BITTE VOLLSTÄNDIG EINGEBEN zwischen den Klammern: {...})
%%
%\ArtDerArbeit{Seminar} % "Bachelor" oder "Projekt" wählen
\TitelDerArbeit{OS Scheduling Algorithmen}
\AutorDerArbeit{Eric Echtermeyer, Benedikt Prisett, Jannik Völker}
%\Abteilung{<Ihre Abteilung>}
%\Firma{<Ihre Firma>}
\Kurs{WWI-21-DSA}
\Studienrichtung{Data Science}
% TODO 
\Matrikelnummer{5370226, 5709658, 6373947}
\Studiengangsleiter{Prof. Dr.-Ing. habil. Dennis Pfisterer}
\WissBetreuer{Prof. Dr. Maximilian Scherer}
%\FirmenBetreuer{<Ihr(e) Firmenbetreuer(in)>}
\Bearbeitungszeitraum{13.11.2023 -- 07.02.2024}
%\Abgabedatum{11.01.2024}

%%
%% @stud
%%
%% BIBLIOGRAPHY (@stud: Bibliographie-Stil wählen - Position und Indizierung)
%%  Auswahl zwischen: 
%%   NUMERIC Style
%%   IEEE Style
%%   ALPHABETIC Style
%%   HARVARD Style
%%   CHICAGO Style 
%%   (oder eigenen zulässigen Stil wählen) 
%%
%%%%%%%%%%%%%
%% Zitierstil
%%%%%%%%%%%%%
% NUMERIC Style - e. g. [12]
\newcommand{\indextype}{numeric} 
%
% IEEE Style - numeric kind of style 
%\newcommand{\indextype}{ieee} 
%
% ALPHABETIC Style - e. g. [AB12]
%\newcommand{\indextype}{alphabetic} 
%
% HARVARD Style 
%\newcommand{\indextype}{apa} 
%
% CHICAGO Style 
%\newcommand{\indextype}{authoryear}
%
%%%%%%%%%%%%%%%%%%%%%%
%% Position des Zitats
%%%%%%%%%%%%%%%%%%%%%%
\newcommand{\position}{inline} 
%
% (!!) FOOTNOTE POSITION NOT RECOMMENDED IN MINT DOMAIN:
%\newcommand{\position}{footnote}

%% Final: Setzen des Zitierstils und der Zitatposition
\usepackage[backend=biber, autocite=\position, style=\indextype, sorting=none]{biblatex}
\settingBibFootnoteCite

%%
%% Definitionen und Commands
%%
\newcommand{\abs}{\par\vskip 0.2cm\goodbreak\noindent}
\newcommand{\nl}{\par\noindent}
\newcommand{\mcl}[1]{\mathcal{#1}}
\newcommand{\nowrite}[1]{}
\newcommand{\NN}{{\mathbb N}}
\newcommand{\imagedir}{img}

\makeindex

\begin{document}

\setTitlepage

%%%%%%%%%%%%%%%%%%%%%%%%%%%%%%%%%%%
% EHRENWÖRTLICHE ERKLÄRUNG
%
% @stud: ewerkl.tex bearbeiten
%
%\input{ewerkl} 
%\cleardoublepage  
%%%%%%%%%%%%%%%%%%%%%%%%%%%%%%%%%%%

%%%%%%%%%%%%%%%%%%%%%%%%%%%%%%%%%%%
% SPERRVERMERK
%
% @stud: nondisclosurenotice.tex bearbeiten
%
%\input{nondisclosurenotice} 
%\cleardoublepage
%%%%%%%%%%%%%%%%%%%%%%%%%%%%%%%%%%%

%%%%%%%%%%%%%%%%%%%%%%%%%%%%%%%%%%%
%	KURZFASSUNG
%
% @stud: acknowledge.tex bearbeiten
%
%\input{acknowledge}
%\cleardoublepage 
%%%%%%%%%%%%%%%%%%%%%%%%%%%%%%%%%%%

%%%%%%%%%%%%%%%%%%%%%%%%%%%%%%%%%%%
% VERZEICHNISSE und ABSTRACT
%
% @stud: ggf. nicht benötigte Verzeichnisse auskommentieren/löschen
%
\tableofcontents
\cleardoublepage

% Abbildungsverzeichnis
\phantomsection
\addcontentsline{toc}{chapter}{\listfigurename}
\listoffigures
\cleardoublepage

%	Tabellenverzeichnis
%\phantomsection
%\addcontentsline{toc}{chapter}{\listtablename}
%\listoftables
%\cleardoublepage

%	Listingsverzeichnis / Quelltextverzeichnis
%\lstlistoflistings
%\cleardoublepage

% Algorithmenverzeichnis
%\listofalgorithms
%\cleardoublepage

% Abkürzungsverzeichnis
% @stud: acronyms.tex bearbeiten
% !TEX root =  master.tex
\clearpage
\chapter*{Abkürzungsverzeichnis}	
\addcontentsline{toc}{chapter}{Abkürzungsverzeichnis}

\begin{acronym}[XXXXXXX]
	\acro{OS}{Betriebssystem}	
	\acro{CPU}{Central Processing Unit}
\end{acronym} 
\cleardoublepage

%	Kurzfassung / Abstract
% @stud: abstract.tex bearbeiten
%% !TEX root =  master.tex
\chapter*{Kurzfassung (Abstract)}
\addcontentsline{toc}{chapter}{Kurzfassung (Abstract)}

Hier können Sie die Kurzfassung (engl.~Abstract) der Arbeit schreiben. Beachten Sie dabei die Hinweise zum Verfassen der Kurzfassung.


 
%\cleardoublepage

%%%%%%%%%%%%%%%%%%%%%%%%%%%%%%%%%%%%%%%%%%%%%%%%%%%%%%%%%%%%%%%%%%%%%%%%%%%%%%%%%%%%%%%%%%
% KAPITEL UND ANHÄNGE
%
% @stud:
%   - nicht benötigte: auskommentieren/löschen
%   - neue: bei Bedarf hinzufügen mittels input-Kommando an entsprechender Stelle einfügen
%%%%%%%%%%%%%%%%%%%%%%%%%%%%%%%%%%%%%%%%%%%%%%%%%%%%%%%%%%%%%%%%%%%%%%%%%%%%%%%%%%%%%%%%%%

\initializeText
\onehalfspacing

%%%%%%%%%%%%%%%%%%%%%%%%%%%%%%%%%%%
% KAPITEL
%
% @stud: einzelne Kapitel bearbeiten und eigene Kapitel hier einfügen
%
% Einleitung
%\input{introduction}

% mehrere Grundlagen- und Forschungs-Kapitel

% !TEX root =  master.tex

\chapter{Grundlagen der Visualisierung}
Das Ziel dieses Projektes liegt in der Erstellung eines hochwertigen Animationsvideos, um den komplexen Sachverhalt der Scheduling Algorithmen von Betriebssystemen intuitiv darzustellen, sodass interessierte Zuschauer relevante Sachverhalte schnell verstehen können. Um für den Lernenden ein Animationsvideo mit möglichst hoher Qualität erstellen zu können, ist eine Beschäftigung mit theoretischen Grundlagen der Visualisierung und Animation unabdingbar. Daher wird im folgenden auf grundlegende Theorien in einer Literaturrecherche eingegangen, welche direkt mit dem von uns produzierten Animationsvideo eingebracht werden. 

Unsere Herangehensweise ein Animationsvideo zu erstellen, wird durch die Dual-Coding-Theorie von Allan Paivio bestätigt, welche besagt, dass Informationen besser verarbeitet und erinnert werden können, wenn diese sowohl visuell als auch verbal präsentiert werden \autocite{paivio_dual_1991}. Diese Theorie wird bei in unserem Animationsvideo umgesetzt, indem neben den visuellen Darstellungen wie Diagrammen oder animierten Algorithmen diese gleichzeitig stets auch auf der Tonspur erklärt werden. Laut der "Cognitive Load Theory" von John Sweller (1988) ist hierbei aber darauf zu achten, dass die kognitive Belastung des Lernenden berücksichtigt wird. Um effektive Bildungsanimationen zu erstellen, ist es entscheidend, die Balance zwischen informativen Inhalten und einer überladenen Darstellung zu finden \autocite{sweller_cognitive_2011}. Durch die sinnvolle Anwendung von Animationen, die komplexe Konzepte in einfachere visuelle Elemente zerlegen, kann das Verständnis erleichtert und die kognitive Belastung reduziert werden. Um den Lernenden daher kognitiv nicht zu überlasten, ist es essentiell, dass animierte Darstellungen stets passend zu der Tonspur sind und diese beiden Komponenten nicht voneinander abweichen. Sollte dies doch der Fall sein, kann dies zu Verwirrung oder kognitiver Leistungsüberschreitung des Lernenden kommen, wodurch die vermittelten Inhalte nicht aufgenommen werden können. 

Um das Animationsvideo einprägsamer für den Lernenden gestalten zu können, ist es sinnvoll die Multimedia-Prinzipien nach Richard E. Mayer (2001) anzuwenden. Diese besagen, dass eine angemessene Kombination von Text, Bildern und Animationen das Lernen deutlich verbessern kann \autocite{mayer_multimedia_2002}. Insbesondere das Kontiguitätsprinzip, das besagt, dass verbundene Text- und Bildinformationen zeitlich und räumlich nahe präsentiert werden sollten, ist für die Gestaltung von Bildungsanimationen relevant. Dies wird direkt in unserem Animationsvideo umgesetzt, wie es in Abbildung \ref{fig:screenshot_text} dargestellt ist. Hier werden mit dezent animiertem Text grafische Vorgänge beschrieben, sodass der Lernende sich diesen Text in Ruhe durchlesen kann ohne bei einer kurzen Unaufmerksamkeit das Verständnis für den gesamten Sachverhalt zu verlieren. 

% Screenshot einfügen mit Beschreibung. Etwas mit Grafik & animiertem Text dazu
\begin{figure}[h]
	\centering
	\includegraphics[width=0.8\linewidth]{img/screenshot_text.png} 
	\caption{Die Szene rund um Stelle 00:00min zeigt die Kombination zwischen animierten Bildinformationen mit unterstützendem Text.}
	\label{fig:screenshot_text} 
\end{figure}

Im Kontext von Animationen ist es auch unerlässlich, die Prinzipien der Animation von Disney zu beachten, die von Johnston und Thomas (1981) detailliert beschrieben wurden. Diese Prinzipien, wie "Squash and Stretch" und "Anticipation", sind zwar ursprünglich für die Unterhaltungsanimation entwickelt worden, können jedoch auch in Bildungsanimationen angewendet werden, um Konzepte lebendig und verständlich darzustellen. Beispielsweise kann die Bewegung in einer Animation genutzt werden, um die Dynamik eines mathematischen Prozesses zu verdeutlichen. 

% Screenshot einfügen von "Animation". Wie ein Algorithmus etwas verarbeitet oder so
\begin{figure}[h]
	\centering
	\includegraphics[width=0.8\linewidth]{img/screenshot_animation.png} 
	\caption{Die Szene rund um Stelle 00:00min animierte Darstellung des Round Robin Algorithmus, um komplexe Sachverhalte visuell einfach darzustellen.}
	\label{fig:screenshot_animation} 
\end{figure}

Neben der Wichtigkeit von Animationen ist eine geschickte Auswahl der vorkommenden Farben ebenso essentiell. Laut der Farbtheorie kann die Auswahl von Farben und Kontrasten dabei helfen, wichtige Informationen zu betonen und die visuelle Erfahrung, und somit die vermittelten Inhalte, der Lernenden zu bereichern \autocite{ballard_art_1964}. Auch die Gestaltprinzipien der Wahrnehmung, formuliert von Max Wertheimer (1923), bieten ebenfalls wichtige Einsichten für die Gestaltung von Bildungsanimationen \autocite{wertheimer_untersuchungen_2017}. Diese Prinzipien, wie die Gruppierung von Elementen nach Nähe oder Ähnlichkeit, können genutzt werden, um Beziehungen und Strukturen innerhalb der mathematischen Inhalte zu verdeutlichen. So kann beispielsweise das Prinzip der Nähe dazu verwendet werden, um zu zeigen, wie verschiedene mathematische Elemente miteinander in Beziehung stehen. 
\input{01_1_section_visualiserung}

% !TEX root =  master.tex
\chapter{OS Scheduling Algorithmen}
Das Kernstück eines jeden modernen Betriebssystems ist dessen Fähigkeit eine Vielzahl von Prozessen effizient und effektiv zu verwalten. Diese Prozessverwaltung, auch bekannt als Scheduling, ist eine komplexe Aufgabe, welche darüber entscheidet, welcher Prozess zu welchem Zeitpunkt von der \ac{cpu} verarbeitet wird. Da moderne Betriebssysteme stets eine hohe Anzahl von Hintergrundprozessen bis hin zu anspruchsvollen Anwendungen verarbeiten müssen, ist die Verwendung leistungsfähiger OS Scheduling Algorithmen essentiell. Im Folgenden werden drei unterschiedliche Algorithmen des OS Scheduling vorgestellt, mit aufsteigender Komplexität. Jeder dieser Algorithmen hat eigenen Stärken und Schwächen, die ihn für bestimmte Szenarien und Anforderungen geeignet machen. Von den einfachen, aber grundlegenden Ansätzen wie First Come First Serve bis hin zu komplexeren Strategien wie Multilevel Queue Scheduling, spiegelt die Entwicklung dieser Algorithmen die Fortschritte in der Computertechnologie und unser zunehmendes Verständnis von effizientem Prozessmanagement wider.

% !TEX root =  master.tex

% https://learning.oreilly.com/library/view/operating-system-concepts/9780471694663/ch05.html#basic_concepts
% TODO
% \cite{bkm.161604526420090101}
\section{First-Come-First-Serve}

Einer der grundlegenden Scheduling Algorithmen für Betriebssysteme ist \ac{FCFS}, welcher auch als \ac{FIFO} bekannt ist. \ac{FCFS} verarbeitet eingehende Prozesse in der Reihenfolge ihres Eintreffens, wobei der zuerst ankommende Prozess als erstes prozessiert wird \cite{ANTHONY201621}.
Implementiert wird \ac{FCFS} für gewöhnlich als Warteschlange, aus welcher eingehende Prozesse anschließend sequentiell verarbeitet werden können. 

% \begin{algorithm}
% 	\caption{\ac{FCFS} Scheduling Algorithm}
% 	\label{alg:fcfs}
% 	\begin{algorithmic}[1]
% 		\Procedure{FCFS}{$processes$}
% 		\State $n \gets \text{length}(processes)$
% 		\State Sort $processes$ by arrival time
% 		\For{$i \gets 1$ to $n$}
% 		\If{$i = 1$}
% 		\State $start\_time[i] \gets processes[i].arrival$
% 		\Else
% 		\State $start\_time[i] \gets \max(processes[i].arrival, finish\_time[i-1])$
% 		\EndIf
% 		\State $finish\_time[i] \gets start\_time[i] + processes[i].burst$
% 		\State $waiting\_time[i] \gets start\_time[i] - processes[i].arrival$
% 		\State $turnaround\_time[i] \gets finish\_time[i] - processes[i].arrival$
% 		\EndFor
% 		\State \textbf{return} $start\_time$, $finish\_time$, $waiting\_time$, $turnaround\_time$
% 		\EndProcedure
% 	\end{algorithmic}
% \end{algorithm}

Der Pseudocode in Abbildung \ref{alg:fcfs} implementiert den \ac{FCFS}-Scheduling-Algorithmus für einen Satz von Prozessen. Dabei wird angenommen, dass jeder Prozess Eigenschaften wie Ankunftszeit (arrival) und Ausführungszeit (burst) hat. Der Algorithmus berechnet die Startzeit, Endzeit, Wartezeit und Umlaufzeit für jeden Prozess. 

Der große Vorteil von \ac{FCFS} liegt in dessen Einfachheit und der hieraus resultierenden leichten Implementierbarkeit. Daher wird dieser auch oft in Lehrbüchern im Kontext grundlegender Betriebssystemkonzepte diskutiert. \cite{silberschatz2018pb} % Silberschatz, Galvin und Gagne (2018)
Zudem ist \ac{FCFS} transparent und einfach vorhersehbar, da die Reihenfolge und Bearbeitungsdauer aller Prozesse lediglich von deren Ankunftszeiten abhängig ist. Ein zusätzlicher Vorteil liegt in der fairen Behandlung aller Prozesse, welche ohne Bevorzugung stattfindet, da jeder Prozess in der Reihenfolge seines Eintreffens bearbeitet wird. % (Stallings, 2012)

Nichtsdestotrotz, weist \ac{FCFS} auch signifikante Nachteile auf, weshalb in der Praxis meist von einer alleinigen Nutzung dieses Algorithmus abgesehen wird. Das wesentliche Problem ist nämlich der Convoy-Effekt, bei dem ein langer Prozess, der früh in der Warteschlange erscheint, nachfolgende, kürzere Prozesse verzögert. Diese Situation führt zu einer ineffizienten \ac{CPU}-Auslastung und verlängerten Wartezeiten. % Tanenbaum, Bos (2014) 
Weiterhin berücksichtigt \ac{FCFS} neben der Dauer auch nicht die unterschiedliche Priorität von Prozessen, was besonders nachteilig für interaktive Systeme ist, in denen schnelle Antwortzeiten von höchster Relevanz sind. Diese Mängel machen \ac{FCFS} für viele moderne Anwendungen unpraktikabel. Daher wird im folgenden der OS Scheduling Algorithmus Round Robin näher betrachtet, welcher versucht eine schnellere Antwortzeit zu ermöglichen.

%\textit{Referenzen:}
%\begin{itemize}
%	\item Silberschatz, A., Galvin, P. B., \& Gagne, G. (2018). Operating System Concepts. Wiley.
%	\item Stallings, W. (2012). Operating Systems: Internals and Design Principles. Prentice Hall.
%	\item Tanenbaum, A. S., \& Bos, H. (2014). Modern Operating Systems. Pearson.
%\end{itemize}
% !TEX root =  master.tex

\section{Round Robin}

Der Round Robin Scheduling Algorithmus ist ein weit verbreitetes OS-Scheduling-Verfahren, welches vor allem für seine Balance zwischen Fairness und Reaktionsfähigkeit bekannt ist. Es ist eines der ältesten aber immer noch sehr weit verbreitetsten Verfahren \cite[S.158]{Tanenbaum.2024}.  
Beim Round-Robin-Algorithmus werden die anstehenden Prozesse wie zuvor beim \ac{FCFS}-Prinzip zunächst in einer Warteschlange gesammelt. Zum Abarbeiten der Aufgaben wird jedem Prozess ein festes Zeitintervall, auch Zeit-Quantum oder Slice genannt, zugewiesen. Dieses Quantum ist in der Regel zwischen 10 und 100 Millisekunden lang und stellt die Dauer dar, welche die CPU nacheinander für einen Prozess aufbringt \cite[S.209]{Silberschatz.2019}. Der erste Prozess wird somit mit der Länge des Quantums bearbeitet und ist am Ende dieses entweder abgeschlossen oder muss unterbrochen werden. Der Algorithmus ist somit preemptive, da die Bearbeitung eines Prozesses unterbrochen werden kann. Ist dies der Fall, wird der Prozess an das Ende der Warteschlange gestellt und der nächste Prozess in der Warteschlange, ebenfalls mit dem gleichen Quantum, bearbeitet. Es entsteht somit ein zirkulares Verfahren, bei dem die Prozesse nacheinander in gleich großen Schritten bearbeitet werden. Dies gewährleistet, dass alle Prozesse regelmäßige \ac{cpu}-Zeit erhalten und kein Prozess andere blockiert, wie es bei \ac{FCFS} der Fall ist. Der Grobalgorithmus in \ref{alg:rr} stellt vereinfacht diesen Ablauf des Round-Robin-Scheduling Verfahrens übersichtlich dar.

\begin{algorithm} 
\caption{Round Robin Scheduling} \label{alg:rr}
\begin{algorithmic}[1]
	\State \textbf{Initialize:} Zeitquantum $q$, Prozesswarteschlange $Q$
	\While{Prozesse existieren in $Q$}
	\State Prozess $P \gets Q$.dequeue()
	\State Weise CPU $P$ für Zeit $\min(P.\text{Restlaufzeit}, q)$ zu
	\If{$P$.\text{Restlaufzeit} $> 0$}
	\State $Q$.enqueue($P$) \Comment{$P$ ist nicht fertig, zurück in die Warteschlange}
	\EndIf
	\If{$Q$ ist leer}
	\State Warte auf neue Prozesse
	\EndIf
	\EndWhile
\end{algorithmic}
\end{algorithm}

Vor allem die Quantum-Länge spielt beim Round-Robin-Scheduling eine entscheidende Rolle, da der Wechsel zwischen den Prozessen, auch als Kontextwechsel bezeichnet, jedes Mal etwas Zeit beansprucht. Ein zu kurzes Quantum führt zu häufigen Kontextwechseln und erhöhtem Overhead, während ein zu langes Quantum die Reaktionszeiten verlängert, da Prozesse länger auf CPU-Zeit warten müssen. Ein angemessenes Quantum minimiert diesen Overhead und hält die Reaktionszeiten kurz. Ein Bereich von 20–50 Millisekunden für das Quantum ist oft ein guter Kompromiss, um die negativen Effekte von Kontextwechseln zu begrenzen und eine effiziente Prozessbearbeitung zu gewährleisten \cite[S.158 f.]{Tanenbaum.2024}.

Insgesamt bietet der Round-Robin-Algorithmus eine ausgewogene Lösung für das Scheduling-Problem, insbesondere in Umgebungen, bei welchen Fairness und schnelle Antwortzeiten gefordert sind. Seine Einfachheit und Effizienz machen ihn zu einer beliebten Wahl in vielen Betriebssystemen.

% !TEX root =  master.tex

\section{Multilevel Queue Scheduling}
Die Prozesse, welche ein modernes Betriebssystem verarbeitet kann in unterschiedliche Kategorien unterteilt werden. So gibt es beispielsweise interaktive Prozesse, bei welchen eine schnelle Antwortzeit essentiell ist, und Hintergrundprozesse, welche nicht direkt abgearbeitet werden müssen. Für gewöhnlich haben interaktive Prozesse daher eine höhere Priorität und müssen daher schneller \ac{CPU}-Zeit zugeteilt bekommen. \ac{MLQ} Scheduling ist ein fortgeschrittener Scheduling-Algorithmus, welcher versucht diese Prozesse mit unterschiedlichen Prozessen effizient zu verwalten. Bei \ac{MLQ} wird die Prozesswarteschlange in mehrere separate Warteschlangen aufgeteilt, wobei jede Warteschlange eine eigene Prioritätsebene besitzt. Eingehende Prozesse werden nun basierend auf ihrer Priorität in die jeweilig zuständige Warteschlange eingeteilt. Jede dieser Warteschlangen hat nun einen eigenen Scheduling-Algorithmus, um eine differenzierte Behandlung der Prozesse zu ermöglichen. % Silberschatz, Galvin und Gagne (2018)
Bei der Verarbeitung der Prozesse wird nun stets die Warteschlange mit der höheren Priorität zuerst abgearbeitet, bis diese vollständig entleert wurde. Anschließend fängt die Verarbeitung der nächst-geringeren Prioritätsstufe an. Sofern ein Prozess mit höherer Priorität nun eintreffen sollte, wird die Verarbeitung der Warteschlange mit geringerer Priorität pausiert. 

\begin{algorithm}
	\caption{Multilevel Queue Scheduling Algorithmus mit \ac{FCFS} and Round Robin}
	\label{alg:mlq}
	\begin{algorithmic}[1]
		\Procedure{MLQ}{$queues$, $processes$, $quantum$}
		\State Assign each process to a queue based on its priority or category
		\For{each $queue$ in $queues$}
		\If{queue is for interactive tasks}
		\State Apply Round Robin Scheduling (Refer to Round Robin Algorithm) with quantum $quantum$
		\State Execute interactive tasks in $queue$ using RR
		\ElsIf{queue is for background tasks}
		\State Apply First-Come, First-Served Scheduling (Refer to FCFS Algorithm)
		\State Execute background tasks in $queue$ using FCFS
		\EndIf
		\EndFor
		\State \textbf{return} scheduling results for each process
		\EndProcedure
	\end{algorithmic}
\end{algorithm}

Der Pseudocode aus Abbildung \ref{alg:mlq} beschreibt, wie bei \ac{MLQ} verschiedene Warteschlangen für interaktive und Hintergrundtasks verwendet werden, wobei für interaktive Tasks der Round Robin und für Hintergrundtasks der \ac{FCFS} angewendet wird.

Der zentrale Vorteil von \ac{MLQ} liegt in dessen Flexibilität und Effizienz bei der Behandlung verschiedener Prozesstypen. Beispielsweise können Systemprozesse, interaktive Prozesse und Batch-Prozesse in verschiedenen Warteschlangen mit entsprechenden Prioritäten und Scheduling-Strategien verwaltet werden. Hierdurch wird eine bessere Anpassung an die Anforderungen spezifischer Prozesstypen erreicht, was zu einer verbesserten Gesamtleistung des Systems führt. % Tanenbaum und Bos (2014) 

Ein Nachteil von \ac{MLQ} liegt allerdings in seiner Komplexität, sowohl in der Implementierung als auch im Management. Die korrekte Einordnung von Prozessen in Warteschlangen und die Auswahl geeigneter Scheduling-Algorithmen für jede Warteschlange erfordern sorgfältige Planung und ständige Anpassung. Eine nachteilige Auswahl und Konfiguration dieser Algorithmen kann zu einem erhöhten Overhead führen und die Systemeffizienz negativ beeinträchtigen. % Stallings (2012)

Trotz dieser Herausforderungen ist \ac{MLQ} ein beliebter Scheduling Algorithmus in Betriebssystemen, insbesondere dort, wo eine Vielzahl unterschiedlicher Prozesse und Anforderungen effizient verwaltet werden muss.

%\textit{Referenzen:}
%\begin{itemize}
%	\item Silberschatz, A., Galvin, P. B., \& Gagne, G. (2018). Operating System Concepts. Wiley.
%	\item Tanenbaum, A. S., \& Bos, H. (2014). Modern Operating Systems. Pearson.
%	\item Stallings, W. (2012). Operating Systems: Internals and Design Principles. Prentice Hall.
%\end{itemize}

% MLQ Text Benedikt eine erste Version (kann ggf. mit dem oberen zusammengeführt werden)
Aufbauen auf anderen Prozess-Schedulern, wie bspw. den zuvor beschriebenen First Come First Serve oder Round-Robin Prinzip gibt es weitere Verfahren, welche durch eine erweiterte Komplexität beabsichtigen zuvor entwickelte Prinzipien und Stärken zu vereinen und Schwächen zu umgehen. Eines dieser Verfahren ist das Multilevel Queue Scheduling. Es handelt sich hierbei um einen Algorithmus, bei welchem Prozess abhängig ihrer Eigenschaften in verschiedene Kategorien eingeteilt und anschließend entsprechend mit unterschiedlicher Priorität bearbeitet werden. Es soll somit, durch das Priorisieren von zeitkritischen Aufgaben und der dynamischen Ressourcenzuweisung, dem Ziel einer gerechten und effizienten Prozessverwaltung weiter nähergekommen werden.
Beim Multilevel Queue Scheduling Verfahren werden zunächst die aktuell offenen, zu bearbeitenden Prozesse unterschiedlichen Warteschlangen dauerhaft zugewiesen. Diese Einteilung kann auf Grundlage unterschiedlicher Kriterien geschehen und ausschlaggebend können Speichergröße, Prozesspriorität oder der Prozesstyp sein. Eine einfache Aufteilung ist beispielweise die Zuordnung in Vordergrundaktivitäten und Hintergrundaktivitäten sodass die erstere Gruppe interaktive Prozess umfasst welche zeitnah abgearbeitet werden müssen und die zweitere Gruppe eher statische Prozesse die ggf. auch länger für die Bearbeitung benötigen. Jede der so geformten Warteschlagen kann unabhängig, auf unterschiedliche Art und Weise bearbeitet werden und verfügt über einen eigenen Scheduling-Algorithmus. So ist es bspw. üblich die Warteschlange für Vordergrundaktivitäten nach dem Round Robin Prinzip abgearbeitet werden, währen bei der zweiten Warteschlange der Hintergrundaktivitäten das First Come First Serve Prinzip Anwendung findet. Dies hat den Hintergrund, dass ? %TODO: Eigenschaften RR, FCFS nochmal angucken und erlärung beenden 
Neben der unterschiedlichen Verfahren die innerhalb der Wartschlangen stattfinden gibt es ein einfaches Scheduling-Verfahren zur Verwaltung der Warteschlangen untereinander. Dieses ist für gewöhnlich mit einer festen Priorisierung und Präemptiv implementiert. Das bedeutet, dass Warteschlangen absolute Prioritäten über anderen haben und eine höher priorisierte Wartschlange zunächst vollständig abgearbeitet wird, gleichzeitig die Bearbeitung einer niedrig priorisieren Warteschlange aber zugunsten neuer Prozesse in einer anderen Schlange unterbrochen werden kann. 
In dem einfachen Beispiel mit zwei Warteschlangen würden daher zunächst die Vordergrundaktivitäten nach dem Round Robin Verfahren abgearbeitet und sobald diese Wartschlange leer ist die Hintergrundaktivitäten mittels First Come First Serve erledigt werden. Tritt während der Bearbeitung der Hintergrundaktivitäten ein Vordergrundprozess auf wird diese Bearbeitung solange unterbrochen, bis wieder alle Vordergrundaktivitäten abgearbeitet sind.
Das Verfahren ist hierbei nicht wie in diesem Beispiel auf zwei Warteschlangen beschränkt, sondern kann um eine Vielzahl an Schlangen für verschiedene Prozesseigenschaften erweitert werden, wie in Abbildung X dargestellt.
%TODO: ggf. Abbildung nachbauen.
Das Multilevel Queue Scheduling Verfahren bietet aufgrund seiner erweiterten Komplexität gegenüber herkömmlichen deutlich einfacheren Verfahren einige Vorteile aber auch Nachteile. So ist positiv zu bemerken, dass die Reaktionszeit des Systems durch die effizientere Ressourcenallokation reduziert werden kann und die Nutzererfahrung durch die schnellere Abarbeitung von interaktiven Prozessen verbessert wird. Desweiterem kann mit diesem Verfahren der Durchsatz gesteigert werden und das System ggf. auch Prozesse unterschiedlicher Schlangen gleichzeitig ausführen, welches zu der Effizienz des Systems beiträgt. Negativ zu betrachten ist hingegen die erhöhte Komplexität beim Design eines effizienten Systems, sowie der zusätzliche Arbeitsaufwand zur Verwaltung der mehreren Warteschlangen untereinander, welches die Performance des Systems wiederum lindern kann. Ein weiteres Problem ist das „Verhungern“ von Prozess in einer Warteschlange mit niedrigere Priorität, welches auftreten kann wenn zu viele, große Prozesse in anderen Schlangen zuerst abgearbeitet werden müssen.
Um diesem Nachteilen des „Verhungerns“ von Prozessen entgegenzuwirken gibt es Weiterentwicklungen der einfachen Multilevel Queue wie beispielsweise das heute verbreiterte Multilevel Feedback Queue Verfahren.
%TODO: ggf. MLFQ kurz erklären

% Hauptquelle: https://drive.uqu.edu.sa/_/mskhayat/files/MySubjects/2017SS%20Operating%20Systems/Abraham%20Silberschatz-Operating%20System%20Concepts%20(9th,2012_12).pdf S275




% !TEX root =  master.tex
\chapter{Performance Analyse}
Die richtige Auswahl der zuvor vorgestellten Scheduling Algorithmen ist ein essenzieller Bestandteil, um die Gesamtleistung und Reaktionsfähigkeit eines Systems zu optimieren. Die zentrale Herausforderung hierbei ist die begrenzte Verfügbarkeit der \ac{CPU}, da diese zu einem Zeitpunkt stets nur einen Prozess ausführen kann. \ac{OS} Scheduling Algorithmen versuchen den Einsatz der \ac{CPU} zu optimieren, indem diese entscheiden welcher Prozess als nächstes bearbeitet werden soll. Die Auswahl des am besten geeigneten Algorithmus muss daher gewährleistet werden \autocite{goel_comparative_2013}. Dieses Kapitel beschäftigt sich daher mit den wichtigsten Metriken zur Auswahl der Algorithmen, um im anschließenden Abschnitt \ac{FCFS}, Round Robin und \ac{MLQ} basierend auf Simulationsergebnissen miteinander zu vergleichen.
% !TEX root =  master.tex

\section{Metriken}
Um unterschiedliche Scheduling Algorithmen wie \ac{FCFS}, Round Robin oder \ac{MLQ} bezüglich ihrer Performance und somit ihrer Eignung für Anwendungsgebiete vergleichen zu können, ist es möglich diverse Metriken zu verwenden. Die Auswahl ist abhängig vom jeweiligen Anwendungsgebiet, beispielsweise wird in interaktiven Systemen eine geringe Antwortzeit priorisiert, während bei Batch-Processing-Systemen eine niedrige Ausführungszeit von höherer Relevanz ist \autocite{thombare_efficient_2016}. Im Folgenden werden häufig genutzte Metriken erklärt, welche im Anschluss für den quantitativen Vergleich verwendet werden.


\textbf{\ac{CPU}-Auslastung (\ac{CPU} Utilization)}: Diese Metrik gibt an, wie effektiv die \ac{CPU} genutzt wird. Eine hohe \ac{CPU}-Auslastung bedeutet, dass die \ac{CPU} aktiv an Prozessen arbeitet, was ein Indikator auf eine effiziente Nutzung der Ressourcen ist \autocite{pemasinghe_comparison_2022}. Die folgende Formel \ref{met:cpu} berechnet den prozentualen Wert der \ac{CPU}-Auslastung, zu welcher die \ac{CPU} Prozesse bearbeitet, bezogen auf die gesamte Beobachtungszeit.
\begin{equation}
	\textit{CPU-Auslastung} = \frac{\textit{Gesamtaktive Zeit}}{\textit{Gesamtbeobachtungszeit}} \times 100
	\label{met:cpu}
\end{equation}


\textbf{Durchsatz (Throughput)}: Der Durchsatz misst die Anzahl der Prozesse, welche in einer zuvor definierten Zeiteinheit vollständig abgearbeitet werden. Hierbei bedeutet ein höherer Durchsatz eine effizientere Verarbeitung von Prozessen durch das System  \autocite{pemasinghe_comparison_2022}. Bei vollständiger \ac{CPU}-Auslastung ist der Durchsatz bei unterschiedlichen Algorithmen konstant, sofern von einem möglichen Overhead durch Context Switches abgesehen wird. Die Formel \ref{met:throughput} berechnet den Durchsatz als Anzahl der Prozesse, die pro Zeiteinheit abgeschlossen werden.
\begin{equation}
	\textit{Durchsatz} = \frac{\textit{Anzahl\ der\ Prozesse}}{\textit{Zeiteinheit}}
	\label{met:throughput}
\end{equation}


\textbf{Durchschnittliche Antwortzeit (Response Time)}: Die Antwortzeit berechnet die Zeit vom Beginn eines Prozesses bis zur ersten Antwort. Diese Metrik ist besonders wichtig in interaktiven Systemen, wo eine schnelle Reaktion auf Benutzereingaben erforderlich ist  \autocite{pemasinghe_comparison_2022}. In Formel \ref{met:responsetime} wird die Berechnung der Wartezeit dargestellt, welche die durchschnittliche Antwortzeit über alle \( n \) Prozesse ist.
\begin{equation}
	\textit{Durchschnittliche Antwortzeit} = \frac{\sum_{i=1}^{n} (\textit{Startzeit}_{i} - \textit{Ankunftszeit}_{i})}{n}
	\label{met:responsetime}
\end{equation}


\textbf{Durchschnittliche Wartezeit (Waiting Time)}: Die Wartezeit eines Prozesses ist die Gesamtzeit, welche dieser in der Warteschlange verbringt, bevor er Zugang zur \ac{CPU} erhält  \autocite{pemasinghe_comparison_2022}. Niedrigere Wartezeiten sind meist erstrebenswert, da es sich hierdurch für gewöhnlich um ein reaktionsfähigeres System handelt. In folgender Formel \ref{met:waitingtime} ist \( n \) die Anzahl der Prozesse, und die Wartezeit wird als Durchschnitt der Zeit berechnet, die jeder Prozess vom Ankunftszeitpunkt bis zum Start der Ausführung wartet.
\begin{equation}
\textit{Durchschnittliche Wartezeit} = \frac{\sum_{i=1}^{n} ((\textit{Abschlusszeit}_{i} - \textit{Ankunftszeit}_{i}) - \textit{Bearbeitungszeit}_{i})}{n}
\label{met:waitingtime}
\end{equation}


\textbf{Durchschnittliche Umlaufzeit (Turnaround Time)}: Bei der Umlaufzeit handelt es sich um die gesamte Dauer vom Start eines Prozesses bis zu dessen Abschluss \autocite{pemasinghe_comparison_2022}. Es wird somit sowohl die Wartezeit, als auch die Bearbeitungszeit berücksichtigt, weshalb ein umfassenderer Vergleich bezüglich der Effizienz von Scheduling Algorithmen möglich ist. Die Ausführungszeit wird als durchschnittliche Gesamtzeit berechnet, die ein Prozess vom Ankunftszeitpunkt bis zum Abschluss benötigt.
\begin{equation}
	\textit{Durchschnittliche Umlaufzeit} = \frac{\sum_{i=1}^{n} (\textit{Abschlusszeit}_{i} - \textit{Ankunftszeit}_{i})}{n}
	\label{met:turnaroundtime}
\end{equation}


\textbf{Fairness}: Die Metrik der Fairness misst, wie gleichmäßig die \ac{CPU}-Zeit zwischen den verschiedenen Prozessen aufgeteilt wird. Ein fairer Scheduling-Algorithmus stellt sicher, dass kein Prozess übermäßig bevorzugt oder benachteiligt wird \autocite{haldar_fairness_1991}. Es handelt sich hierbei um eine qualitative Metrik, weshalb es in der Literatur keine festgelegte Berechnung hierfür gibt. Stattdessen wird die Fairness oft durch die Analyse der Verteilung von den Wartezeiten über die verschiedenen Prozesse hinweg bewertet, beispielsweise durch die Berechnung der Standardabweichung. Folgende Formel \ref{met:fairness} stellt die beispielhafte Berechnung hiervon dar, wobei dies nicht als universeller Weg angesehen wird. 
\begin{equation}
	\textit{Fairness} = \sqrt{\frac{\sum_{i=1}^{n} (\textit{Wartezeit Prozess}_{i} - \textit{durchschnittliche Wartezeit})^2}{n}}
	\label{met:fairness}
\end{equation}

% !TEX root =  master.tex
\section{Simulationsergebnisse}
Für eine Simulation der drei zuvor beschriebenen \ac{OS} Scheduling Algorithmen ist die Erstellung eines abzuarbeitenden Datensatzes mit Prozessen essentiell. Mithilfe einer randomisierten Generierung der Prozesse wird versucht eine möglichst hohe Vielzahl von Szenarien abzudecken, um die Leistungsfähigkeit der Algorithmen unter verschiedenen Bedingungen zu testen. Jeder Prozess besitzt hierbei die eine Ankunftszeit, Bearbeitungsdauer und eine geringe oder hohe Priorisierung, wie zu sehen im Quelltext \ref{lst:sequence-diagram-process}.

\begin{lstlisting}[caption={Prozess mit Attributen in Python implementiert}, label={lst:sequence-diagram-process}]
class Process:
  def __init__(
    self, id: int, arrival_time: int, burst_time: int, priority: str = "low"
  ) -> None:
	self.id = id
	self.arrival_time = arrival_time
	self.burst_time = burst_time
	self.priority = priority
\end{lstlisting}

Für die randomisierte Generierung der Prozesseigenschaften wird zunächst die Ankunftszeit \textit{arrival\_time} der Prozesse bestimmt. Um eine realitätsnahe Simulation zu gewährleisten, wird die Ankunftszeit so variiert, dass sowohl gleichmäßig verteilte als auch in Clustern ankommende Prozesse im Datensatz vorhanden sind. Diese Variation wird durch Anpassung der Ankunftszeit jedes nachfolgenden Prozesses erreicht, wobei eine zufällige Abweichung berücksichtigt wird, um die Clusterbildung zu simulieren.

Des Weiteren wird die Bearbeitungsdauer \textit{burst\_time} basierend auf einer Normalverteilung bestimmt, um realistische Variationen im Datensatz zu beinhalten. Hierbei sind die Parameter des Mittelwertes und der Standardabweichung zu wählen. Ein weiterer relevanter Parameter ist die Verteilung der Priorität \textit{priority} innerhalb des Datensatzes. Da zu hohe oder geringe Werte zu einer Vernachlässigung der Vorteile von Round Robin und \ac{MLQ} Scheduling führen, ist es wichtig hierbei eine Balance zu finden. Tabelle \ref{tab:process_dataset_parameters} zeigt die Wahl der unterschiedlichen Prozessparameter auf. Obwohl diese Parameter mit der Intention einer möglichst neutralen Verteilung der Prozesse sorgfältig festgelegt wurden, ist es wichtig zu betonen, dass die Prozessspezifikationen je nach Anwendungsfall abweichen können.

\begin{table}[htbp]
	\centering
	\begin{tabular}{ll}
		\toprule
		Parameter                              & Beschreibung                                   \\
		\midrule
		Anzahl der Prozesse                    & 100 (für eine umfangreiche Evaluation)         \\
		Durchschnittliche Burst-Zeit           & 250 ms (durchschnittliche Ausführungszeit)     \\
		Standardabweichung der Burst-Zeit      & 600 ms (für breite Streuung der Anforderungen) \\
		Prozentsatz der Hochprioritätsprozesse & 20\% (ein Fünftel aller Prozesse)              \\
		Variation der Ankunftszeit             & 100 ms (für unterschiedliche Ankunftsmuster)   \\
		\bottomrule
	\end{tabular}
	\caption{Parameter für die Erstellung des Prozess-Datensatzes}
	\label{tab:process_dataset_parameters}
\end{table}

Nach der Erstellung des Datensatzes werden diese Prozesse nun von \ac{FCFS}, Round Robin und \ac{MLQ} scheduled und abgearbeitet. Die zuvor erläuterten Metriken werden berechnet und in Tabelle \ref{tab:scheduling_comparison} dargestellt.

\begin{table}[htbp]
	\centering
	\begin{tabular}{lccc}
		\toprule
		Metrik                       & FCFS      & Round Robin & MLQ       \\
		\midrule
		Durchschnittliche Wartezeit  & 43.65 ms  & 31.37 ms    & 43.02 ms  \\
		Durchschnittliche Umlaufzeit & 339.73 ms & 327.45 ms   & 339.10 ms \\
		Unfairness-Index             & 148.68    & 91.22       & 147.14    \\
		Kontextwechsel               & 100.00    & 678.00      & 165.00    \\
		\bottomrule
	\end{tabular}
	\caption{Vergleich der Scheduling-Algorithmen}
	\label{tab:scheduling_comparison}
\end{table}

Bei Betrachtung der Ergebnisse wird deutlich, dass Round Robin eine geringere durchschnittliche Wartezeit im Vergleich zu \ac{FCFS} und \ac{MLQ} aufweisen kann. Dies ist der Fall, da mithilfe der Einteilung in Quanten Prozesse gleichmäßiger verarbeitet und Bottlenecks vermieden werden. Obwohl \ac{MLQ} ebenfalls diese Funktionsweise für die Prozesse mit hoher Priorität beinhaltet, sind diese Effekte nur marginal sichtbar. Ähnlich verhält es sich mit der durchschnittlichen Umlaufzeit, welche neben der Wartezeit auch die Ausführungszeit beinhaltet. Auch beim Unfairness-Index, bei welchem geringe Werte zu priorisieren sind, da diese auf eine faire Abarbeitung der Prozesse hinweisen, wird deutlich, dass Round Robin die Prozesse gleichmäßiger abarbeiten kann. Diese Vorteile in der Wartezeit, Umlaufzeit und der Fairness haben allerdings den Preis einer höheren Anzahl an Kontextwechseln. Round Robin kann diese Vorteile nämlich nur erzielen, indem regelmäßig zwischen den Prozessen rotiert wird. Diese Kontextwechsel führen zu Overhead des gesamten Systems. Da die zeitliche Größe dieser Kontextwechsel sehr systemabhängig ist und von zahlreichen Faktoren abhängt, wurde diese nicht simuliert. Grundlegend können hierbei Werte zwischen wenigen Mikrosekunden bis hin zu etwa einer Mikrosekunde angenommen werden.

Die Simulation zeigt auf, dass \ac{FCFS}, Round Robin und \ac{MLQ} jeweils individuelle Stärken und Schwächen besitzen. Während \ac{FCFS} die Anzahl der Kontextwechsel minimiert und hierdurch ein effizientes System darstellt, bieten Round Robin und \ac{MLQ} deutliche Vorteile für interaktive Systeme, bei welchen eine schnelle Reaktion auf Prozesse hoher Priorität entscheidend ist. Die Auswahl der jeweiligen Scheduling Algorithmen kann daher nicht pauschalisiert werden, sondern hängt stark vom spezifischen Anwendungsfall ab.

% !TEX root =  master.tex
\chapter{Anwendungsgebiete}
Innerhalb dieses Kapitels werden wird auf konkrete Anwendungsgebiete der zuvor genannten OS-Scheduling Algorithmen eingegangen. Da diese Algorithmen auch außerhalb von Betriebssystemen einen zentrale Rolle spielen, werden auch Beispiele solcher Anwendungsgebiete aufgenommen.

% !TEX root =  master.tex

\section{Innerhalb von Betriebssystemen}
Ein häufiges Anwendungsgebiet von \ac{FCFS} ist für Aufgaben der Batch-Verarbeitung, da hierbei der Convoy-Effekt nicht entscheidend ist. Ein klassisches Beispiel ist das Druckmanagement in frühen Betriebssystemen, wo Druckaufträge in der Reihenfolge ihres Eintreffens abgearbeitet werden.  %\textit{Referenz:} Silberschatz, Galvin, and Gagne. \textit{Operating System Concepts}. 9th ed., John Wiley \& Sons, 2012.

Round Robin hingegen ist ein weit verbreiteter Scheduling Algorithmus in zeitkritischen Betriebssystemen. Ein Beispiel hierfür ist das Thread-Scheduling in Unix-basierten Systemen, bei dem jedem Thread eine feste Zeitscheibe zugeteilt wird. %\textit{Referenz:} Tanenbaum, Andrew S., and Bos, Herbert. \textit{Modern Operating Systems}. 4th ed., Pearson, 2014.

\ac{MLQ} wird in modernen Betriebssystemen wie Linux angewendet, um Prozesse basierend auf ihrer Priorität zu ordnen, wobei systemkritische Prozesse höher priorisiert werden als Benutzerprozesse. % \textit{Referenz:} Love, Robert. \textit{Linux Kernel Development}. 3rd ed., Addison-Wesley Professional, 2010.

In modernen Betriebssystemen ist der Aufbau der Prozessverwaltung und Anwendung von Scheduling Algorithmen ein komplexes Thema, welches in der Praxis eine Vielzahl von Anforderungen erfüllen muss. Die hier vorgestellten Algorithmen sind nur ein kleiner Ausschnitt der Vielzahl von Scheduling Algorithmen, die in modernen Betriebssystemen zum Einsatz kommen.

\subsubsection{Windows}
Windows verwendet beispielsweise 7 Prioritätsstufen.
Prozesse können sich selbst die folgenden Stufen zuweisen:
\begin{multicols}{3}
    \begin{itemize}[noitemsep]
        \item IDLE
        \item BELOW NORMAL
        \item NORMAL
        \item ABOVE NORMAL
        \item HIGH
        \item REALTIME
    \end{itemize}
\end{multicols}

Innerhalb eines Prozesses können die einzelnen Threads dann jeweils 7 Prioritätsebenen haben, die die Threads untereinander sortieren:
\begin{multicols}{3}
    \begin{itemize}[noitemsep]
        \item IDLE
        \item LOWEST
        \item BELOW NORMAL
        \item NORMAL
        \item ABOVE NORMAL
        \item HIGHEST
        \item CRITICAL
    \end{itemize}
\end{multicols}

Windows verwendet beide Prioritätszuweisungen um dem Thread eine Basispriorität zu geben, die zwischen 0 und 31 liegt \autocite{KarlBridgeMicrosoft.2023}.

\subsubsection{MacOS}
Mit der Einführung der neuen M-Serie Prozessoren unterteilt Apple die Prozessorkerne in Performance- und Effizienzkerne. Die Performancekerne sind für rechenintensive Aufgaben und die Effizienzkerne für weniger rechenintensive Aufgaben zuständig.
% !TEX root =  master.tex

\section{Außerhalb von Betriebssystemen}
\ac{FCFS} wird in zahlreichen Gebieten verwendet, insbesondere bei Prozessen die eine schwierige Digitalisierbarkeit aufweisen. Beispielsweise wird \ac{FCFS} in der Industrieautomatisierung, insbesondere in der Steuerung von Fertigungsstraßen, verwendet, bei welcher Aufträge in der Reihenfolge ihres Eingangs bearbeitet werden. % \textit{Referenz:} Groover, Mikell P. \textit{Automation, Production Systems, and Computer-Integrated Manufacturing}. 4th ed., Prentice Hall, 2016.
Auch in Bereichen fernab von Computern und Industrie wird \ac{FCFS} in alltäglichen Situationen eingesetzt. Ob die Warteschlange an der Kasse im Supermarkt, die Patientenabfertigung im Krankenhaus in der Notaufnahme oder auch der Essensausgabe in der Kantine. 

Ein bekanntes Anwendungsgebiet von Round Robin ist die Telekommunikation. Hierbei wird Round Robin für die Paketvermittlung und Lastverteilung in Netzwerkroutern eingesetzt. Ein Beispiel ist die Verteilung von Netzwerkbandbreite in Cisco-Routern. %\textit{Referenz:} Kurose, James F., and Ross, Keith W. \textit{Computer Networking: A Top-Down Approach}. 7th ed., Pearson, 2016.
Darüber hinaus wird Round Robin in der Luftfahrtindustrie für die Flugzeugbodenabfertigung eingesetzt, um eine faire und effiziente Zuteilung von Abfertigungsdiensten zu gewährleisten. %\textit{Referenz:} Wells, Alexander T., and Young, Seth B. \textit{Airport Planning \& Management}. 6th ed., McGraw-Hill Education, 2011.

\ac{MLQ} findet beispielsweise Anwendung in Cloud-Computing-Umgebungen wie AWS oder Google Cloud, wo verschiedene Instanzen oder Services in getrennten Warteschlangen basierend auf SLAs verwaltet werden. %\textit{Referenz:} Rhoton, John. \textit{Cloud Computing Explained: Implementation Handbook for Enterprises}. 2nd ed., Recursive Press, 2013.





% Fazit und Ausblick
%\input{conclusion}
%%%%%%%%%%%%%%%%%%%%%%%%%%%%%%%%%%%

%%%%%%%%%%%%%%%%%%%%%%%%%%%%%%%%%%%
% ANHÄNGE
%
% @stud: einzelne Anhänge bearbeiten und eigene Anhänge hier einfügen 
%        die nachfolgenden Zeilen deaktivieren, wenn keine Anhänge verwendet werden
%
\initializeAppendix
% !TEX root =  master.tex
\chapter{Quellcode}
Der vollständige Quellcode ist über folgenden GitHub Link erreichbar: \linebreak https://github.com/echtermeyer/OS-Scheduling-Algorithms-Comparison

%\input{appendix2}
%%%%%%%%%%%%%%%%%%%%%%%%%%%%%%%%%%%

\singlespacing

%%%%%%%%%%%%%%%%%%%%%%%%%%%%%%%%%%%
% LITERATURVERZEICHNIS
% @stud: Literaturverzeichnis in Datei bibliography.bib anpassen. 
%
% Alternative zu Verwendung von \initializeBibliography: Citavi ...
% (dann \initializeBibliography auskommentieren und eigenes LaTex Coding verwenden)
%
\initializeBibliography
%%%%%%%%%%%%%%%%%%%%%%%%%%%%%%%%%%%

%%%%%%%%%%%%%%%%%%%%%%%%%%%%%%%%%%%
% INDEX
% @stud: ggf. Index auskommentieren, wenn nicht benötigt
%
\addcontentsline{toc}{chapter}{Index}
\printindex

\end{document}
