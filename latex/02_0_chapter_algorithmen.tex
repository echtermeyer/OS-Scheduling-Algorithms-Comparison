% !TEX root =  master.tex
\chapter{OS Scheduling Algorithmen}
Das Kernstück eines jeden modernen Betriebssystems is dessen Fähigkeit eine Vielzahl von Prozessen effizient und effektiv zu verwalten. Diese Prozessverwaltung, auch bekannt als Scheduling, ist eine komplexe Aufgabe, welche darüber entscheidet, welcher Prozess zu welchem Zeitpunkt von der \ac{CPU} verarbeitet wird. Da moderne Betriebssysteme stets eine hohe Anzahl von Hintergrundprozessen bis hin zu anspruchsvollen Anwendungen verarbeiten müssen, ist die Verwendung leistungsfähiger OS Scheduling Algorithmen essentiell. Im folgenden werden drei unterschiedliche Algorithmen des OS Scheduling vorgestellt, mit aufsteigender Komplexität. Jeder dieser Algorithmen hat eigenen Stärken und Schwächen, die ihn für bestimmte Szenarien und Anforderungen geeignet machen. Von den einfachen, aber grundlegenden Ansätzen wie First Come First Serve bis hin zu komplexeren Strategien wie Multilevel Queue Scheduling, spiegelt die Entwicklung dieser Algorithmen die Fortschritte in der Computertechnologie und unser zunehmendes Verständnis von effizientem Prozessmanagement wider.
