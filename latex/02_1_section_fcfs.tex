% !TEX root =  master.tex

\section{First-Come-First-Serve}

Einer der grundlegenden Scheduling-Algorithmen für Betriebssysteme ist \ac{FCFS}, auch bekannt als \ac{FIFO}. \ac{FCFS} verarbeitet eingehende Prozesse in der Reihenfolge ihres Eintreffens, wobei der zuerst eingetroffene Prozess zuerst abgearbeitet wird. Ein Prozess wird so lange von der \ac{cpu} bearbeitet, bis er vollständig abgearbeitet ist.
Dadurch wird jedoch verhindert, dass in der Zwischenzeit andere, wichtigere Prozesse  bearbeitet werden \cite[Kapitel 2.3.1.7]{ANTHONY201621}.

In der Regel wird \ac{FCFS} als Warteschlange implementiert, von der aus eingehende Prozesse sequentiell abgearbeitet werden können.

\begin{algorithm}
    \caption{First Come First Serve } \label{alg:fcfs}
    \begin{algorithmic}[1]
        \State \textbf{Initialize:} Prozesswarteschlange $Q$
        \While{True}
        \State Prozess $P \gets Q$.dequeue()
        \State Weise CPU $P$ für Zeit $\text{P.Restlaufzeit}$ zu
        \If{Neuer Prozess $P*$ erreicht CPU}
        \State $Q$.enqueue($P*$)
        \EndIf
        \EndWhile
    \end{algorithmic}
\end{algorithm}


Das Beispiel in Algorithmus \ref{alg:fcfs} implementiert den \ac{FCFS}-Scheduling-Algorithmus. Die Prozesse werden in einer Warteschlange $Q$ gespeichert und sequentiell abgearbeitet. Der Algorithmus weist der \ac{cpu} den ersten Prozess in der Warteschlange zu und bearbeitet diesen, bis die Restlaufzeit des Prozesses abgelaufen ist. Danach wird der nächste Prozess aus der Warteschlange genommen und abgearbeitet. Trifft ein neuer Prozess während der Abarbeitung eines anderen Prozesses ein, so wird dieser in die Warteschlange eingefügt.

Der große Vorteil von \ac{FCFS} liegt in seiner Einfachheit und der daraus resultierenden leichten Implementierbarkeit. Darüber hinaus ist \ac{FCFS} transparent und leicht vorhersehbar, da die Reihenfolge und Bearbeitungsdauer aller Prozesse nur von deren Ankunftszeit abhängt. Ein weiterer Vorteil liegt in der fairen Behandlung aller Prozesse, die ohne Bevorzugung erfolgt, da jeder Prozess in der Reihenfolge seines Eintreffens bearbeitet wird.

Allerdings hat \ac{FCFS} auch erhebliche Nachteile, weshalb in der Praxis meist auf die alleinige Verwendung dieses Algorithmus verzichtet wird. Das Hauptproblem ist der Convoy-Effekt, bei dem ein langer Prozess, der früh in der Warteschlange erscheint, die nachfolgenden kürzeren Prozesse verzögert. Dies führt zu einer ineffizienten \ac{cpu}-Auslastung und zu längeren Wartezeiten.
Darüber hinaus berücksichtigt der \ac{FCFS} neben der Dauer auch nicht die unterschiedliche Priorität von Prozessen, was insbesondere für interaktive Systeme nachteilig ist, in denen schnelle Antwortzeiten von höchster Relevanz sind \cite[Kapitel 5]{Galvin.2004}.

Diese Mängel machen \ac{FCFS} für viele moderne Anwendungen unbrauchbar. Daher wird im folgenden der OS Scheduling Algorithmus Round Robin näher betrachtet, der versucht, eine schnellere Antwortzeit zu ermöglichen.