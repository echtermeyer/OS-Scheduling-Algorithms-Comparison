% !TEX root =  master.tex

\section{Round Robin}

Der Round Robin Scheduling Algorithmus ist ein weit verbreitetes OS-Scheduling-Verfahren, welches für vor allem für seine Balance zwischen Fairness und Reaktionsfähigkeit bekannt ist. Es ist eines der ältesten aber immer noch sehr weit verbreitetsten Verfahren \cite[S.158]{Tanenbaum.2024}.  
Beim Round-Robin-Algorithmus werden die anstehenden Prozesse wie zuvor beim \ac{FCFS}-Prinzip zunächst in einer Warteschlange gesammelt. Zum Abarbeiten der Aufgaben wird jedem Prozess ein festes Zeitintervall, auch Zeit-Quantum oder Slice genannt, zugewiesen. Dieses Quantum ist in der Regel zwischen 10 und 100 Millisekunden lang und stellt die Dauer dar, welche die CPU nacheinander für einen Prozess aufbringt \cite[S.209]{Silberschatz.2019}. Der erste Prozess wird somit mit der Länge des Quantums bearbeitet und ist am Ende dieses entweder abgeschlossen oder muss unterbrochen werden. Der Algorithmus ist somit preemptive, da die Bearbeitung eines Prozesses unterbrochen werden kann. Ist dies der Fall, wird der Prozess an das Ende der Warteschlange gestellt und der nächste Prozess in der Warteschlange, ebenfalls mit dem gleichen Quantum, bearbeitet. Es entsteht somit ein zirkulares Verfahren, bei dem die Prozesse nacheinander in gleich großen Schritten bearbeitet werden. Dies gewährleistet, dass alle Prozesse regelmäßige \ac{CPU}-Zeit erhalten und kein Prozess andere blockiert, wie es bei \ac{FCFS} der Fall ist. Der Grobalgorithmus in \ref{alg:rr} stellt vereinfacht diesen Ablauf des Round-Robin-Scheduling Verfahrens übersichtlich dar.

\begin{algorithm} 
\caption{Round Robin Scheduling} \label{alg:rr}
\begin{algorithmic}[1]
	\State \textbf{Initialize:} Zeitquantum $q$, Prozesswarteschlange $Q$
	\While{Prozesse existieren in $Q$}
	\State Prozess $P \gets Q$.dequeue()
	\State Weise CPU $P$ für Zeit $\min(P.\text{Restlaufzeit}, q)$ zu
	\If{$P$.\text{Restlaufzeit} $> 0$}
	\State $Q$.enqueue($P$) \Comment{$P$ ist nicht fertig, zurück in die Warteschlange}
	\EndIf
	\If{$Q$ ist leer}
	\State Warte auf neue Prozesse
	\EndIf
	\EndWhile
\end{algorithmic}
\end{algorithm}

Vor allem die Quantum-Länge spielt beim Round-Robin-Scheduling eine entscheidende Rolle, da der Wechsel zwischen den Prozessen, auch als Kontextwechsel bezeichnet, jedes Mal etwas Zeit beansprucht. Ein zu kurzes Quantum führt zu häufigen Kontextwechseln und erhöhtem Overhead, während ein zu langes Quantum die Reaktionszeiten verlängert, da Prozesse länger auf CPU-Zeit warten müssen. Ein angemessenes Quantum minimiert diesen Overhead und hält die Reaktionszeiten kurz. Ein Bereich von 20–50 Millisekunden für das Quantum ist oft ein guter Kompromiss, um die negativen Effekte von Kontextwechseln zu begrenzen und eine effiziente Prozessbearbeitung zu gewährleisten \cite[S.158 f.]{Tanenbaum.2024}.

Insgesamt bietet der Round-Robin-Algorithmus eine ausgewogene Lösung für das Scheduling-Problem, insbesondere in Umgebungen, bei welchen Fairness und schnelle Antwortzeiten gefordert sind. Seine Einfachheit und Effizienz machen ihn zu einer beliebten Wahl in vielen Betriebssystemen.
